% Collected by Juntang Oct 4 2024

\subsection{Results}
% Juntang 09/29/2024

\begin{figure}[htbp]
    \centering
    % Indian Pines and KSC Figures
    \begin{minipage}{0.45\textwidth}
        \centering
        \includegraphics[width=\linewidth]{figures/Indian_pines_HARmap.png}
        \caption{Indian Pines HARmap}
        \label{fig:indian_pines_har}
    \end{minipage}\hfill
    \begin{minipage}{0.45\textwidth}
        \centering
        \includegraphics[width=\linewidth]{figures/KSC_HARmap.png}
        \caption{KSC HARmap}
        \label{fig:ksc_har}
    \end{minipage}
\end{figure}

\begin{figure}[htbp]
    \centering
    % Salinas and Salinas-A Figures
    \begin{minipage}{0.45\textwidth}
        \centering
        \includegraphics[width=\linewidth]{figures/Salinas_HARmap.png}
        \caption{Salinas HARmap}
        \label{fig:salinas_har}
    \end{minipage}\hfill
    \begin{minipage}{0.45\textwidth}
        \centering
        \includegraphics[width=\linewidth]{figures/SalinasA_HARmap.png}
        \caption{Salinas-A HARmap}
        \label{fig:salinasa_har}
    \end{minipage}
\end{figure}

The Hamiltonian functions used in this method are defined as follows:

\[
h_a = - \sum_{C \in \Omega} \alpha(C, C)
\]
\[
h_r = \sum_{C \in \Omega} \alpha^2(C, V)
\]


\begin{figure}[htbp]
    \centering
    % Pavia University Figure
    \begin{minipage}{0.45\textwidth}
        \centering
        \includegraphics[width=\linewidth]{figures/PaviaU_HARmap.png}
        \caption{Pavia University HARmap}
        \label{fig:pavia_university_har}
    \end{minipage}\hfill
    \begin{minipage}{0.45\textwidth}
    \end{minipage}
\end{figure}

\newpage
\begin{figure}[htbp]
    \centering
    % WHU-Hi Figures
    \begin{minipage}{0.45\textwidth}
        \centering
        \includegraphics[width=\linewidth]{figures/WHU_Hi_HanChuan_HARmap.png}
        \caption{WHU Hi HanChuan HARmap}
        \label{fig:whu_hanchuan_har}
    \end{minipage}\hfill
    \begin{minipage}{0.45\textwidth}
        \centering
        \includegraphics[width=\linewidth]{figures/WHU_Hi_LongKou_HARmap.png}
        \caption{WHU Hi LongKou HARmap}
        \label{fig:whu_longkou_har}
    \end{minipage}
\end{figure}



\subsection{Datasets}
\begin{table}[h!]
    \centering
    \caption{Summary of hyperspectral datasets}
    \label{tab:datasets}
    \resizebox{0.8\textwidth}{!}{
    \begin{tabular}{lccccccl}
        \hline
        \textbf{Dataset} & \textbf{\#x} & \textbf{\#y} & \textbf{\#\bm{$\lambda$}} & \textbf{\bm{$\lambda$} (nm)} & \textbf{\#C} & \textbf{Res. (m)} & \textbf{Year}\\
        \hline
        Indian Pines & 145 & 145 & 220 & 400--2500 & 16 & 20 & 1992\\
        KSC & 512 & 614 & 176 & 400--2500 & 13 & 18 & 1996\\
        Salinas & 512 & 217 & 204 & 400--2500 & 16 & 3.7 & 1998\\
        Salinas-A & 86 & 83 & 204 & 400--2500 & 6 & 3.7 & 1998\\
        Botswana & 1476 & 256 & 145 & 400--2500 & 14 & 30 & 2001\\
        Pavia & 1096 & 1096 & 102 & 430--860 & 9 & 1.3 & 2003\\
        Pavia University & 610 & 610 & 103 & 430--860 & 9 & 1.3 & 2003\\
        WHU-Hi-HanChuan & 1217 & 303 & 274* & 400--1000 & 16 & 0.109 & 2016\\
        WHU-Hi-HongHu & 940 & 475 & 270* & 400--1000 & 22 & 0.043 & 2017\\
        WHU-Hi-LongKou & 550 & 400 & 270* & 400--1000 & 9 & 0.463 & 2018\\
        BigEarthNet-S2-s & 1200 & 1200 & 13 & 443--2190 & 12 & 10-60 & 2017\\
        \hline
    \end{tabular}
    }
\end{table}

\begin{figure}[htbp]
    \centering
    \includegraphics[width=0.38\linewidth]{output1.png}
    \caption{Teasor's Plot}
    \label{fig:tp}
\end{figure}

\newpage

Indian Pines: \cite{PURR1947}, KSC, Salinas, Salinas-A, Botswana, Pavia and Pavia University: \cite{pavia_datasets}, WHU-Hi-HanChuan, WHU-Hi-HongHu, WHU-Hi-LongKou: \cite{zhongMiniUAVborneHyperspectralRemote2018, huWHUhiUAVborneHyperspectral}, BigEarthNet-S2-s: \cite{clasen2024refinedbigearthnet, hackel2024configilm}.

\newpage