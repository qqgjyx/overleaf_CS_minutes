% meeting minutes: BEER Python prototype by Dimitris

\begin{list}{$-$}{}
\item \dimitris[inline]{Revised the BEN loader:} We now have the correct 
  location coordinates (longitude/latitude) for each tile/pixel, allowing 
  us to stitch the tiles together, even when dealing with potential irregular 
  boundaries. The locations have been verified by overlaying them on 
  OpenStreetMaps and visually comparing the tiles with the RGB images. 
  See Fig.~\ref{fig:ben_overview} for an example subset of the BEN dataset.

\item \dimitris[inline]{Completed the sliding-window prototype of the BEER 
    algorithm:} The sliding window adapts to irregular boundaries and 
  missing tiles. For an illustration, see the animation at the following link:

  \url{https://gitlab.oit.duke.edu/jw853/clustering4hsi/-/raw/main/outdir/10.Oct_10_BEER_Prototype/ben_sliding.gif}
  
\end{list}

\begin{figure}[!htpb]
  \centering
  \includegraphics[width=0.9\linewidth]{figures/ben_overview.png}
  \caption{Overview of a subset of the BigEarthNetV2 dataset (the area
    shown is located in Austria).}
  \label{fig:ben_overview}
\end{figure}

\newpage